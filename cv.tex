\documentclass[9pt]{extarticle}

% \documentclass[10pt]{article} %Sets the default text size to 11pt and class to article.
%------------------------Dimensions--------------------------------------------
\pagenumbering{gobble}% Remove page numbers (and reset to 1)
\topmargin=0.0in %length of margin at the top of the page (1 inch added by default)
\oddsidemargin=0.0in %length of margin on sides for odd pages
\evensidemargin=0in %length of margin on sides for even pages
\textwidth=6.7in %How wide you want your text to be
\marginparwidth=0.5in
\headheight=0pt %1in margins at top and bottom (1 inch is added to this value by default)
\headsep=0pt %Increase to increase white space in between headers and the top of the page
\textheight=9.5in %How tall the text body is allowed to be on each page

\usepackage{xcolor}

\definecolor{linkcolor}{RGB}{105,0,0}

\usepackage[utf8]{inputenc}
\usepackage[colorlinks=true,urlcolor=linkcolor]{hyperref}

\newcommand\tab[1][1cm]{\hspace*{#1}}
\newcommand\smallspace[1][0.23cm]{\hspace*{#1}}
\newcommand\negativespace[1][-0.12cm]{\hspace*{#1}}

%\usepackage[T1]{fontenc}
%\usepackage{libertine}


\begin{document}

\centerline{{\LARGE \bf Maximilian Fuchs MSc.}}
\centerline{\small \href{https://maxfuchs.net}{maxfuchs.net} \raisebox{0.25ex}{\tiny$\bullet$}  \href{mailto:maxfuchs@pm.me}{maxfuchs@pm.me}}


\noindent %Prevents the following text from being indented
\\\\\\
\vspace*{-6pt}
{\negativespace \Large \bf Professional Experience}\\
\line(1,0){485}
\\
\noindent

\noindent
{\bf Senior Software Engineer}  \hfill \textit{From August 2022} \\
{\bf Software Engineer}, \textit{employee at \href{https://www.3m.com/3M/en_US/health-information-systems-us/support/international/}{3M Health Information Systems}}  \hfill \textit{June 2020 - August 2022} 
\begin{itemize}
\setlength\itemsep{0.05em}
\item Developed and maintained a medical  \hfill Germany, Remote \\
natural language understanding platform in Java and XML
\item Implemented MLOps workflows in AWS, including building, training, and deploying large models
\item Collaborated with cross-functional teams to understand business requirements and translate them into technical solutions 
\item Mentored junior team members and provided technical guidance to help them succeed in their roles 
\item Acquired and prepared large medical datasets for use in training and evaluating deep learning models 
% \item Designed and implemented new functionalities for a medical natural language understanding platform in Java and XML \\
\end{itemize}

\noindent
{\bf Research Assistant}, \textit{working student at \href{https://www.de-cix.net/}{DE-CIX}}   \hfill \textit{March 2018 -- May 2020}
\begin{itemize}
\setlength\itemsep{0.05em}
\item Implemented and tested SDN components for production in Python and C \hfill Frankfurt, Germany
\item Wrote technical specification for federal cloud project “Gaia X”
\item Performed Data Science tasks for research papers and business intelligence \\
(SDN, System Engineering, Internet Infrastructure) \\
\end{itemize}

\noindent
{\bf Research Assistant}, \textit{working student at \href{https://www.informatik.tu-darmstadt.de/ukp/ukp_home/index.en.jsp}}{TU Darmstadt, UKP Lab}  \hfill \textit{October 2017 -- February 2018}
\begin{itemize}
\setlength\itemsep{0.05em}
    \item Deployed machine learning models in a production environment \hfill Darmstadt, Germany
    % \item Developed a web-based annotation tool for NLP tasks (Java, Spring) \\
    \item Contributed to software development projects as part of a scrum team
    \item Tutored NLP programming classes for graduate students \\
    (Java, Python, Java Spring) \\
\end{itemize}

\noindent
{\bf Research Assistant}, \textit{working student at \href{https://www.informatik.tu-darmstadt.de/telekooperation/telecooperation_group/index.en.jsp}}{Telecooperation Lab, TU Darmstadt}  \hfill \textit{February 2015 -- March 2017}
\begin{itemize}
\setlength\itemsep{0.05em}
    \item Developed a microservice based video-streaming application \hfill Darmstadt, Germany
    \item Set up virtual network and computing infrastructure 
    \item Developed and compared performance of schedulers for container orchestration using docker swarm and go
    \item Implemented and designed GUI elements for Microsoft Pixelsense \\
    (C#, Docker, Vagrant, Go, Docker Swarm, Bash) \\
\end{itemize}
%\noindent
%\noindent

\noindent
{\bf Software development intern}, \textit{part-time at sofasession GmbH}  \hfill \textit{February 2015 -- June 2015}
\begin{itemize}
    \setlength\itemsep{0.05em}
    \item Implemented signal processing components in C++ \hfill Vienna, Austria
    \item Tested and debugged software components for a real-time audio processing system \\
    (C++, Git, JavaScript, HTML/CSS) \\
\end{itemize}

\noindent %Prevents the following text from being indented
\\
\vspace*{-6pt}
{\negativespace \Large \bf Education}\\
\line(1,0){485}\\
\\
\noindent
{\bf Master of Science - Internet- and Web-based Systems} \hfill \textit{July 2020} \\ 
\textit{Technische Universität Darmstadt (Thesis: Rnd of a cloud video object tracker [Vue.js, Docker, RabbitMQ, OpenCV])}\\\\
\noindent
{\bf Bachelor of Science - Network Engineering} \hfill \textit{March 2016} \\
\textit{University of Applied Sciences Carinthia, Austria (Thesis: Rnd of a Shopfront-Editor [backbonejs])}\\
\\
%\noindent
\\\\
\vspace*{-6pt}
{\negativespace \Large \bf Skills}\\
\line(1,0){485}\\
\\
\noindent
{\bf Responsive Web-Apps } (\textit{Typescript, React, Vue, Svelte, Electron, WebRTC, WebGL}) \\
{\bf Distributed Web-Services } (\textit{Node.js, NestJS, Python, AMQP, CQRS, Redis, SQL/NoSQL, data warehouse}) \\
{\bf API Design }(\textit{REST, GraphQL, openAPI, oAuth2}) \\
{\bf Testing }(\textit{Jest, Puppeteer, device farms}) \\
{\bf Deployment }(\textit{GCP, AWS, Docker/Kubernetes, CI/CD, serverless, edge computing}) \\
{\bf Misc }(\textit{cli-apps, data migration, UI/UX design, mentoring}) \\\\
\noindent
{\bf Advocate of:} Functional/reactive programming, QA/test-automation, microservice architecture,\\ 
 \tab \tab \smallspace event sourcing, small + frequent releases


\end{document}
